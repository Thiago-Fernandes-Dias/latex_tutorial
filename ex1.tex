\documentclass[12pt, a4paper]{article}

\usepackage[utf8]{inputenc}
\usepackage[brazil]{babel}
\usepackage{setspace}

\setlength{\parskip}{6pt}

\setlength{\parindent}{1.5cm}

\onehalfspacing 

\title{\textbf{\underline{Um breve relato sobre Ramanujan}}\\
\textsc{(minicurso introdução ao \LaTeX})}

\date{\today}

\begin{document}

\maketitle

\begin{abstract}
\indent 
Um misterioso matemático, nascido em 1887 em Erode, na Índia, criou
teoremas surpreendentes. Sem formação acadêmica, realizou contribuições
substanciais nas áreas da análise matemática, teoria dos números, séries
infinitas, frações continuadas, \ldots Esse documento conta um pouco de 
sua história.
\end{abstract}

\section{Introdução}

\noindent 
Srinivasa Aiyangar Ramanujan (Erode, 22 de dezembro de 1887 -- Kumbakonam, 26 
de abril de 1920), foi um matemático indiano.  Seus estudos nunca foram 
efetivamente publicados.  O que existe da sua obra são essencialmente fórmulas 
e expressões matemáticas isoladas e rascunhos manuscritos.

Ramanujan era um matemático com um modo de trabalhar especial.  Embora não
tivesse o conceito de demonstração enraizado, a verdade é que possuía uma
intuição admirável.  São muitas as contribuições do matemático, sendo ele
considerado um gênio marcante para os estudos das ciências exatas.

Ao procurar por matemáticos que pudessem entender seu trabalho, em 1913 ele
começou a trocar cartas com G. H. Hardy\footnote{Godfrey Harold Hardy foi um 
matemático inglês, conhecido por seu trabalho em teoria dos números e análise 
matemática.} da Universidade de Cambridge, Inglaterra.  Eventualmente Hardy 
trouxe Ramanujan para o Trinity College de
Cambridge, onde os dois deram início a uma frutuosa relação de trabalho.

Com saúde muito frágil por vários anos, ele morreu em 1920, em Kumbakonam, na
Índia.  Sua história é relatada no filme intitulado \textit{The Man Who Knew 
Infinity} (''O homem que viu o infinito''), de 2015, que por sua vez é baseado 
em um livro de 1991, de mesmo nome.

O restante desse documento está dividido da seguinte forma. Na Seção 2
apresentamos alguns dos eventos mais importantes sobre a vida de Ramanujan. Na
Seção 3 apresentamos alguns dos resultados obtidos pelo matemático. Na Seção 4
mostramos como a história do Ramanujan já foi inserida na cultura popular. Por
fim, na Seção 5 indicamos algumas de suas publicações.

Extra para a introdução:

Sobre G. H. Hardy: Godfrey Harold Hardy foi um matemático inglês, conhecido
por seu trabalho em teoria dos números e análise matemática.

Sobre o filme: Esse filme teve orçamento de \$10 milhões.

\section{Vida}

\noindent 
Srinivasa Ramanujan tem uma história de vida bastante interessante e cheia de
detalhes. A seguir tentamos resumir os eventos mais importantes:

\begin{description}
    \item[1887] Nasceu em Erode, Tamil Nadu, India, em 22 de dezembro;
    \begin{itemize}
        \item Filho de K. Srinivasa Iyengar \& Komalatammal.
    \end{itemize}
    \item[1906] Entrou na Universidade Pachaiyappa, em Madras, mas saiu sem 
                completar os estudos;
    \item[1911] Publicou primeiro artigo sobre Números de Bernoulli;
    \item[1913] Escreve a primeira carta a G. H. Hardy;
    \item[1914] E. H. Neville conhece Ramanujan em Madras e o convence a ir para 
                Cambridge;
    \item[1916] Consegue grau de bacharel na Universidade de Cambridge;
    \item[1917] É constantemente hospitalizado para tratamentos;
    \item[1918] Se torna \textit{Fellow of the Royal Society};
    \item[1919] Eleito para bolsa de estudos no Trinity College em Cambridge;
    \item[1920] De volta à Índia, com piora na saúde, morre em 26 de Abril de
                1920.
\end{description}

Veja na Figura 1 uma foto de Ramajunam.

\end{document}
